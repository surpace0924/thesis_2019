\chapter{開発手法}
 この章ではモデルベース開発という開発手法について概説し,提案システムの開発プロセスについて説明する.

\section{モデルベース開発}
モデルベース開発(Model Based Design / Development MBD)とは,シミュレーション可能なモデルを用いるソフトウェア開発手法のことを指す.制御系MBDでは,制御器および制御対象,またはその一部をモデルで表現し,机上シミュレーション/リアルタイムシミュレーションにより制御アルゴリズムの開発・検証を行う.このモデル化とシミュレーションは,ハードウェアが利用できない場合(システム開発初期)には特に有益である.さらに,コード生成ツールを用いて,制御器モデルから実際の制御器(マイコン等)に組み込む制御用Cプログラムを作成することも可能となる.コード生成により,時間を節約し手作業でのコーディングによるエラーを防ぐことができる\cite{MBD_Simulink}.

\section{MATLAB/Simulink}
 本研究では,信号処理アルゴリズムの開発にMATLAB/Simulinkを使用した.MATLABはThe MathWorks社が開発している数値解析ソフトウェアであり,その中で使うプログラミング言語の名称でもある\cite{MATLAB_MathWorks}.また,SimulinkはMATLABと統合化されたモデルベース開発のためのソフトウェアである.モデルにMATLABアルゴリズムを組み込み,シミュレーションの結果をMATLABにエクスポートして詳細な解析を行うことができる\cite{Simulink_MathWorks}.


% \section{提案システムの開発手法}
% 立振子制御 C API の開発は、具体的には図 1-3 に示されている開発プロセスに沿っておこないました.開発プロセスを大別すると次の 4 つに分けることができます.
% 制御対象のデータ収集: 制御対象である 2 輪倒立振子ロボットの物理的振る舞いを解析するための運動方程式の導出および 2 輪倒立振子ロボット物理モデルの作成
% 制御アルゴリズム設計(ラピッドプロトタイピング): 2 輪倒立振子ロボットを制御するための制御器の設計/評価。Simulink 上に構築された Embedded Coder Robot NXT という LEGO MINDSTORMS NXT 用モデルベース開発環境を用いた、2 輪倒立振子ロボット物理モデルと制御器のシミュレーションおよび制御器の実機検証(ラピッドプロトタイピング)
% 実装モデル設計: 制御アルゴリズム設計で開発された倒立振子制御アルゴリズムをベースに実装用モデルを設計
% 組み込み用コード生成: 実装モデルから Real-Time Workshop Embedded Coder という C コード生成ツールを用いて、倒立振子制御 C API を自動生成

