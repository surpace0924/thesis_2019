\chapter{緒言}
\section{研究背景}
 自動車の走行時に生じる振動は乗車している人に負担となり,乗り心地の快適性を損ねる.特に上下方向の振動を軽減する装置としてはバネとダンパを用いるサスペンションが広く用いられている.その中でも,アクティブサスペンションは自動車の揺れを検知し,車内で感じる揺れを抑えるように電子制御することが可能である.\\
 このアクティブサスペンション制御の研究は古くから行われており,1989年には市販車に装備されるまでにその技術は進歩している\cite{active_suspension}.しかし,その多くはストロークセンサや加速度センサーがバネ下の動きを検知してから作動するもので,車両が段差に侵入した際の最初の衝撃を完全に吸収できない.そこで,車載センサを用いて路面の状態を測定し,車体に衝撃が加わるタイミングでサスペンションを調整することにより,既存のアクティブサスペンション技術では吸収しきれない最初の衝撃を軽減できると考える.\\

\section{研究目的}
 本研究では,このアクティブサスペンションの動作判定の前段として,路面の段差検知システムの実現を目的とする.\\
 提案するシステムはRADARを用いて路面を常に監視し,車両前方の段差の有無を判別する.システム構築にはMATLAB/Simulinkを用いたモデルベース開発を採用し,ソフトウェア設計の効率向上を図った.なお,提案した手法の有効性をシミュレーションを通して検証した.\\

\section{本論文の構成}
 本論文は7章より構成され,第1章は緒論であり,本文は第2章から始まる.\\
 2章では現在主に使用されている車載センサについて説明し,3章でRADARの原理の説明とLiDARとの比較を行う.4章ではモデルベース開発について説明を行い,5章で実際に提案するシステムについて述べる.6章で検証結果について説明する.最後に,結論を7章に示す.\\
