\chapter{結論}
 本稿の目的は,歩行者との衝突を回避するために,自動減速システムを実現することである.提案システムは,現在の位置から歩行者の将来の位置を予測し,衝突確率を検出することが可能である.また,走行抵抗とモデル化誤差を補償することができるMPCとコントローラを,この論文で提案されている.\\
 まず,歩行者の速度は,カルマンフィルターにより,現在の歩行者の位置から推定される.将来の歩行者の予想位置は,歩行者の速度から予測され,制動力の指令値は,MPCによって計算される.ここでは,MPCは外乱とパラメータ誤差の影響を強く受けるため,制御性能を向上させるために,制動力のフィードバックループが提案されている.さらに,提案されたシステムには,走行抵抗の推定量が含まれている.これにより,歩行者の将来の位置を予測し,衝突の危険性を検出することが可能である.さらに,駆動抵抗を補償し,制動力を制御することにより,外乱感度を低減しながらシステムの追従性を向上させることができる.\\
 提案手法の有効性は,シミュレーションと実験により検証されている.このシステムを活用して,事故の数を減らすことが期待される.