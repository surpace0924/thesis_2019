{\small
\begin{thebibliography}{9}
  % 1
  \bibitem{active_suspension} 武馬修一・趙在成・神田亮・梶野英紀・土田久輔・十津憲司・大谷佳史. 電動アクティブサスペンションアクチェータの開発. 自動車技術会論文集. 2008, 39, 5, p. 13.

  % 2
  \bibitem{jidountenlab_sensor} 自動運転LAB. 【最新版】自動運転の最重要コアセンサーまとめ LiDAR、ミリ波レーダ、カメラ. \url{https://jidounten-lab.com/y_2520}, (参照:2020-01-24)

  \bibitem{denso_sensor} 松ヶ谷和沖. 自動運転を支えるセンシング技術. Denso technical review. 2016, 21, p. 13-21.

  \bibitem{MITTechnologyReview_sensor} MIT Technology Review. Self-Driving Cars’ Spinning-Laser Problem. \url{https://www.technologyreview.com/s/603885/autonomous-cars-lidar-sensors/?set=603886}, (参照:2020-01-24)

  \bibitem{zmp_sensor} ZMP.Autonomous Driving(自動運転)の制御に使われるセンサについて. \url{https://www.zmp.co.jp/knowledge/ad_top/dev/sensor}, (参照:2020-01-24)

  \bibitem{cleantechnica_sensor} CleanTechnica. Tesla \& Google Disagree About LIDAR — Which Is Right?. \url{https://cleantechnica.com/2016/07/29/tesla-google-disagree-lidar-right/}, (参照:2020-01-24)


  % 3
  \bibitem{RADAR_book}梶原 昭博 (2019). ミリ波レーダー技術と設計. 科学情報出版.

  \bibitem{soumu_RadioWaves} 総務省. 周波数帯ごとの主な用途と電波の特徴. \url{https://www.tele.soumu.go.jp/j/adm/freq/search/myuse/summary/}, (参照:2020-02-06)

  \bibitem{feature_RadioWaves} 藤村契二. 車載用ミリ波レーダの実用化技術. 電気学会論文誌. 1998, 118, p. 292.

  \bibitem{lidar_datasheet} YDLIDAR. YDLIDAR X4 Datasheet. \url{http://www.ydlidar.com/Public/upload/files/2019-12-18/YDLIDAR%20X4%20Datasheet.pdf}, (参照:2020-01-18)

  % 4
  \bibitem{MBD_Simulink} MathWorks. Model-Based Design with Simulink. \url{https://jp.mathworks.com/help/simulink/gs/model-based-design.html}, (参照:2020-02-10)

\end{thebibliography}
}