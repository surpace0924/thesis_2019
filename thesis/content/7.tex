\chapter{結言}

\section{本論文のまとめ}
 本研究では車両前面に取り付けたRADARを用いて路面の状態を計測し,段差の有無を検出するシステムを開発した.提案した手法は,RADARから得られた路面までの距離を微分し,その絶対値が閾値を超えた時に段差検知と判定するものである.本論文での実験および考察において得られた知見を以下に示す.
\begin{itemize}
    \item シミュレーションにおいて提案した手法を検証したところ,15m/s$^2$以上の加速度が発生するような段差の検出を行うことができた.
    \item 提案したアルゴリズムには6.3節に挙げられるような問題がまだ残っており,実用化に向けては改良の余地がある.
\end{itemize}

 この手法を改良し,走行中の車両の前方の段差を事前に検知することが可能となれば,車両が段差に乗り上げるのと同じタイミングでアクティブサスペンションを動作させ,段差走行時の乗り心地を改善できると考える.

\section{今後の展望}
 本研究では路面の状態を計測するためにシステムをRADARを用いて構築し,検証を行った.しかしながら,6.3節に挙げられるような問題がまだ残っており,検知精度の向上のためにもアルゴリズムの改良は必須である.アルゴリズムの改良の際はRADARのみで得られるデータだけでは限界があるため,カメラやLiDAR,その他センサを併用し,センサフュージョンを行う手法が考えられる.また,現状のシステムは段差の有無を出力するものにとどまっており,実際の車両の運動制御への応用を視野にいれた場合,段差の大きさの特定についても考える必要がある.
