\chapter{車載センサの種類}
 この章では自動車の運転支援システムや自動運転に活用されているセンサについて説明する.
\section{カメラ}
 一般的に自動運転やADAS向けのカメラは車内にあるルームミラーの裏側などに配置されており、車両の進行方向を向いている。その場合、前方カメラはウインドガラスを挟んで前方の画像を撮影し,画像処理用プロセッサが撮影した画像・映像の解析をリアルタイムで行う.この過程を経て,車両の前方に車両や障害物や人がいるかを検知することができる。\cite{jidountenlab_sensor}\\
 カメラは種々の対象物を検出・認識することができ,対象物に応じて複数の用途に利用することができる.夜間の運転時に,対向車のヘッドライトを検出することにより,自車のヘッドライトのハイ/ロービームの切り替えを自動で行う機能,速度制限の標識を認識して速度警告を行う機能,道路上の白線を認識し,その位置から自車のレーン逸脱を警報する機能,前方の車両や歩行者を検知して,衝突の危険がある際に警報を出し,緊急時には自動でブレーキを掛ける機能,等々,様々な用途に用いることができる.\\
 他にも,このような運転支援の用途だけではなく,自動運転を行う際にも周囲の車両や歩行者,交通標識,道路上の白線などを検出・認識できる重要なセンサである.\\
 なお,前方を2台のカメラ(ステレオカメラ)を用いると,2台のカメラの映像の視差から物体までの距離を推測することが可能になる.\cite{denso_sensor}\\
 一方、カメラで画像・映像を撮影するということは、基本的には人の目で見るという仕組みと類似の原理であることから、夜間や逆光に加え、濃霧、豪雨、豪雪などの悪天候の場合は検出能力が低下することが課題の一つとされている。\cite{jidountenlab_sensor}\\

\section{LiDAR}
LiDAR は,赤外線のレーザ光をパルス状に照射し,物体に反射されて帰ってくるまでの時間から距離を計測するセンサである.動作原理がレーダと類似しているため,別名レーザレーダとも呼ばれる.細く絞ったレーザ光を可動ミラーによって方向を変えてスキャンすることで物体の方位も検出することができる.このようなタイプのセンサをスキャン LIDAR と呼ぶ.LIDAR は,ミリ波レーダに比べてさらに波長の短い電磁波である,赤外光を使っているため,検出の際の空間分解能が高いことが特長である.この特長を生かし車の進路の安全な場所の検出に使うことができる.ただし,赤外光を用いるため,豪雨,豪雪,
霧などの悪天候時に検出性能が低下するという短所がある.\cite{denso_sensor}また,LiDARは高価で屋外で使用できる性能のものになると1個あたり数百万円もする.さらに,LiDARは測定時に物理的に回転する.耐久性の観点から量販車に実装することは現状のままでは少し厳しい.\cite{MITTechnologyReview_sensor}

\section{RADAR}
ミリ波レーダは,ミリ波と呼ばれる非常に波長の短い電波を照射し,物体に反射されて帰ってくる電波を検出することにより,物体までの距離と方向を検出するセンサである.現状利用されているミリ波レーダは,前方検出用には 76GHz,後方や側方検出用には 24GHz が多く用いられている.Fig. 6 に前方検出用の 76GHz ミリ波レーダ(量産品)の構造を示す.ミリ波が透過する筐体カバー(レドーム)の中に,ミリ波を送受信するアンテナ,ミリ波の信号を処理する RF 回路,受信信号をデジタル化して演算処理

\section{総括}
 以上で紹介した,カメラ,ミリ波レーダ,LIDAR の三種類のセンサは,いずれも長所と短所がある.カメラによる検出は,物体の識別が可能であり,車両や歩行者など自動車を安全に走行させるうえで重要な物体を,他の物体と区別して検出することができる.また,道路標識や路面上のレーンマークの認識も可能である.しかしながら,カメラの画像は人間の目で見る画像と同じ原理に基づくものであり,夜間や逆光など光源が不適切なシーンや,濃霧,豪雨,豪雪などの悪天候のシーンでは,人間と同じく検出能力が低下する. これに対しミリ波レーダは,自らの発する電波を利用した検出のため,光源や天候に左右されず良好な検出特性を維持できる.また対象物体までの距離を正確に計測できる特長もある.しかしながら,検出の際の空間分解能が他のセンサに比べて劣るため,物体の識別は困難であり,また段ボール箱や発泡スチロールなど,電波の反射率の低い物体の検出が難しいという課題がある.LIDAR は,赤外線のレーザ光を用いるため,電波の反射率が低い物体も検出できる.特に段ボール箱,木材,発泡スチロールなど,路上散乱物として走行の妨げになる物体も検出可能である.またスキャン LIDAR では高い空間分解能で距離と方位を検出できるため,物体検出だけでなく,それらの間のフリースペースの検出も可能である.ただし,赤外光を用いるため,豪雨,豪雪,霧などの悪天候時に検出性能が低下するという短所がある. 以上の状況を踏まえ,運転支援や自動運転の際には,上記のセンサのうち一種類だけを用いるのではなく,複Fig. 8 Function and mechanism of LIDARFig. 9 Free-space detection by processing LIDAR data19数のセンサを組み合わせて検出の信頼性を高めることが望ましい.例えば前方の障害物を検知して,警報や非常ブレーキを掛ける運転支援の用途には,前方カメラとミリ波レーダとを組み合わせ,検出の信頼性を向上させ,昼夜・天候による検出性能の変動を抑制するとともに,誤検知を抑制する工夫を行っている. また,非常ブレーキの場合には,物体を検出した場合は車を停止させれば安全性を担保できるが,自動運転を実現する際には,前方に安全に車が走行できるフリースペースを探し,車を停止させずにその領域を通って走行を継続させる必要がある.このため,自動運転の際には,フリースペース検知性能に優れたスキャン LIDAR を追加することが検討されている.

% cleantechnica_sensor
% http://sensait.jp/281/
