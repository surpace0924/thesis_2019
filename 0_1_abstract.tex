\begin{abstract}

    {\Large \bf 概要\par}
    \vspace{3mm}
     本研究は,車両と歩行者との衝突を回避するための自動減速システムの実現を目的としている.提案システムは現在位置から歩行者の将来位置を予測し,衝突確率を検出することができる.さらに,駆動抵抗やモデリング誤差を補償することができるモデル予測制御を採用したコントローラも提案されている.提案した方法の有効性をシミュレーションと実験を通して検証した.\par
    \vspace{1mm}
    {\bf キーワード:}
    衝突被害軽減ブレーキ(AEBS),モデル予測制御(MPC),駆動力オブザーバー(DFOB),カルマンフィルタ\par
    \vspace{7mm}

    {\Large \bf Abstract\par}
    \vspace{3mm}
    The purpose of this study is to realize an automatic deceleration system to avoid collision between vehicles and pedestrians. The proposed system can predict the pedestrian's future position from the current position and detect the collision probability. In addition, controllers using model predictive control that can compensate for driving resistance and modeling errors have been proposed. The effectiveness of the proposed method was verified through simulations and experiments.
    \par
    \vspace{1mm}
    {\bf Keyword:}
    Autonomous Emergency Braking System (AEBS), Model Predictive Control (MPC), Driving Force Observer (DFOB), Kalman filter\par

\end{abstract}