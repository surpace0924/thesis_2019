\chapter{モデルベース開発}
 この章では提案システムでの開発に用いた,モデルベース開発という開発手法の全般について概説する.

\section{モデルベース開発}
モデルベース開発(Model Based Design / Development、MBD と略されます)とは、シミュレーション可能なモデルを用いるソフトウェア開発手法です。制御系 MBD では、制御器および制御対象、またはその一部をモデルで表現し、机上シミュレーション/リアルタイムシミュレーションにより制御アルゴリズムの開発・検証を行います。リアルタイムシミュレーションとは、制御系の一部を実機、その他をリアルタイムシミュレータ上で動作するモデル生成コードとし、実時間での動作検証を行うシミュレーション技術のことです。さらに、Real-Time Workshop® Embedded Coder 等の C コード生成ツールを用いて、制御器モデルから実際の制御器(マイコン等)に組み込む制御用 C プログラムを作成することができます。図 1-1はThe MathWorks社のMATLABという製品を用いた場合の制御系MBDの概念図です。

