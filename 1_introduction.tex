\chapter{序論}
 車両制御技術は,情報処理の先端技術として急速に発展している.たとえば,急ブレーキ時にホイールロックを防止するアンチロックブレーキシステム(ABS),滑りやすい路面でタイヤがスリップするのを抑制する電子安定性制御(ESC)などがあげられる.また,各車輪の駆動力を独立して制御して車両の挙動を安定させるダイレクトヨーモーメント制御(DYC)と,駆動力を制御することで所望のスリップ率を維持するスリップ率制御など.このような制御技術の発展により,自動車はこれまで以上に安全かつ安定して走行することが可能になってきた.運転支援システムは,交通事故件数の削減に大きく貢献し,さらなる発展が期待されている.\cite{harris}\\
 近年,交通事故を減らす技術として自律型緊急ブレーキシステム(AEBS)が注目されている.AEBSの概要:車両に搭載されたカメラやレーダーが他の車両や前方の歩行者などの障害物を検出した場合,相対位置と相対速度から衝突の危険性を判断する.すると,システムは車両を減速させる.自動車製造業者の調査によると\cite{susan},事故の合計数は,非装備車と比較して61%減少し,後端事故の数は84%減少した.それはAEBSの重要性と有用性を証明している.\\
 しかしながら,現在の商用車のAEBSは,歩行者の現在位置に基づいて衝突の危険性を検出しており,歩行者の動きを考慮していないという問題がある.歩行者や障害物が車両の前方に存在する場合,システムはそれに対処することができ,衝突事故を回避することが可能である.一方,歩行者が横から歩くと,衝突の瞬間まで歩行者が車両の前に位置していないため,現在のAEBSは機能しない可能性がある.歩行者と車両との衝突事故は道路を横断する間にしばしば起こるので,歩行者運動に対応するAEBSは交通事故を減らすための重要なシステムである.\\
 歩行者の動きに対応して,歩行者の将来の位置を予測し,それに応じて制動力を制御する必要がある.したがって,有限時間における車両の位置がシステムの指令値となるように,時間領域で車両を制御する必要がある.\\
 そこで,時間領域での制御器設計法としてモデル予測制御(MPC)が注目されている.MPCは制約条件付きリアルタイム最適制御と呼ぶことができ,将来のシステム挙動を予測し,評価関数を最小化するように現在の入力値を決定する.MPCは二次計画問題を解決する必要があり,計算負荷が大きいため,プロセス制御の分野のようにサンプリング時間の長いシステムで広く使用されている.しかし,近年,計算機の計算速度が向上してきており,サンプリング時間の短いシステムへの適用が可能になってきている.MPCのもう一つの問題は,開ループコントローラである.制御性能はモデル精度に大きく依存し,外乱の感度が高いという欠点がある.\\
 外乱抑制と目標値への追従性を両立させるためには,外乱を推定しフィードフォワード制御で補償する必要がある.外乱とは,転がり抵抗,空気抵抗,坂道走行時に発生する勾配抵抗などの走行抵抗のことである.しかしながら,駆動力制御に関するこれまでの研究のほとんどは,駆動抵抗が非常に小さいと仮定し無視している.サンプリング時間が短いシステムの場合,フィードバックループによる駆動抵抗の影響を少なくすることができる.しかしながら,MPCはPID制御等よりもサンプリング時間が長く,外乱の影響が顕著になる.さらに,外乱はMPCにおける予測に使用されるモデルの精度を低下させるので,自動車の予測された挙動は実際のものから逸脱する.これでは所望の結果が得られない.したがって,MPCを駆動力制御に適用するためには,駆動抵抗の推定と補償を考慮する必要がある.\\
 一般に,人間はさまざまな道路環境や歩行者に基づいて危険を予測しながら運転する.ここでは,人間などの危険性を予測しながら車両を制御することを目的としており,歩行者の動きに対応するためのAEBSの改良につながる.この目的を達成するために,歩行者の将来の位置を予測するためのアルゴリズムの構築と運転抵抗を補償することができる時間領域におけるコントローラの設計について述べる.\\
 本論文では,車両に搭載されたカメラなどのセンシング機器から得られる歩行者の位置情報を想定している.提案手法の概要は以下の通りである.まず,歩行者速度は,現在の歩行者の位置からカルマンフィルタによって推定される.将来の歩行者の予測位置は,一定期間の歩行者速度の値から予測される.衝突の可能性がある場合は,車両の予測位置と比較してMPCにより制動力の指令値を算出する. MPCはモデルの精度に依存するため,外乱とパラメータ誤差の影響を強く受けるという欠点がある.そこで,制御性能を向上させるために制動力のフィードバックループを導入し,外乱である駆動抵抗の推定量を提案システムに含める.システムのコントローラはMPCと制動力フィードバック制御のカスケードコントローラである.提案した方法を利用することにより,歩行者の将来の位置を予測し,衝突の危険性を検出することが可能である.さらに,駆動抵抗を補償して制動力を制御することで,外乱感度を低下させながらシステムの追従性を向上させることができる.\\
 この論文は以下のように構成されている. 2章では提案手法のモデル化について説明し,3章で制御系の設計について述べる.4章と5章ではそれぞれシミュレーションと実験について説明する.最後に,結論は6章に示す.\\