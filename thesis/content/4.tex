\chapter{開発手法}
 この章ではモデルベース開発という開発手法について概説し,提案システムの開発プロセスについて説明する.

\section{モデルベース開発}
モデルベース開発(Model Based Design / Development MBD)とは,シミュレーション可能なモデルを用いるソフトウェア開発手法のことを指す.制御系MBDでは,制御器および制御対象,またはその一部をモデルで表現し,机上シミュレーション/リアルタイムシミュレーションにより制御アルゴリズムの開発・検証を行う.このモデル化とシミュレーションは,ハードウェアが利用できない場合(システム開発初期)には特に有益である.さらに,コード生成ツールを用いて,制御器モデルから実際の制御器(マイコン等)に組み込む制御用Cプログラムを作成することも可能となる.コード生成により,時間を節約し手作業でのコーディングによるエラーを防ぐことができる\cite{MBD_Simulink}.

\section{MATLAB/Simulink}
 本研究では,信号処理アルゴリズムの開発にMATLAB/Simulinkを使用した.MATLABはThe MathWorks社が開発している数値解析ソフトウェアであり,その中で使うプログラミング言語の名称でもある\cite{MATLAB_MathWorks}.また,SimulinkはMATLABと統合化されたモデルベース開発のためのソフトウェアである.モデルにMATLABアルゴリズムを組み込み,シミュレーションの結果をMATLABにエクスポートして詳細な解析を行うことができる\cite{Simulink_MathWorks}.

\section{提案システムの開発手法}
提案システムの開発は,具体的には以下の開発プロセスに沿って行った.
\begin{enumerate}
    \item \textbf{制御対象のデータ収集}\\
        RADAR信号処理アルゴリズムを構築する上で,シミュレータに入力するためのテストデータを実機から収集した.
    \item \textbf{信号処理アルゴリズム設計}\\
        RADAR信号から所望のデータ(路面の段差の有無)を算出する信号処理器をSimulink上に構築し,シミュレーションを行った.
    \item \textbf{アルゴリズムの検証}\\
        事前に収集したRADAR信号のデータを解析し,アルゴリズムの有効性を検証した.
\end{enumerate}
