\documentclass[autodetect-engine, dvipdfmx-if-dvi, ja=standard, 12pt]{bxjsreport}

\usepackage{graphicx}        %図を表示するのに必要
\usepackage{color}           %jpgなどを表示するのに必要
\usepackage{amsmath,amssymb} %数学記号を出すのに必要
\usepackage{type1cm}         % fontsizeのエラー回避
\usepackage{setspace}
\usepackage{cases}
\usepackage{here}
\usepackage{fancyhdr}
\usepackage{ascmac}
\usepackage{url}
% \usepackage{titlesec}
\usepackage{subfigure}

\usepackage{listings}
%ここからソースコードの表示に関する設定
\lstset{
    basicstyle={\ttfamily\small}, %書体の指定
    % frame=tRBl, %フレームの指定
    frame={tb},
    breaklines=true, %行が長くなった場合の改行
    linewidth=12cm, %フレームの横幅
    lineskip=-0.5ex, %行間の調整
    numbers=left,
    tabsize=2 %Tabを何文字幅にするかの指定
}
% \lstset{
%     frame=single,
%     numbers=left,
%     tabsize=2
% }

\parindent = 0pt  % 行頭の字下げをしない
\setstretch{1.4}  % 行間を広めにとる

% 各章,節などタイトルの大きさを変更
% \titleformat*{\section}{\Huge\bfseries}
% \titleformat*{\subsection}{\Large\bfseries}

% 式の番号を(senction_num.num)のようにする
\makeatletter
\@addtoreset{equation}{chapter}
\def\theequation{\thechapter.\arabic{equation}}
\makeatother

% 呼び出したページのページ番号を消す
\newcommand{\deletePageNum}{
    \thispagestyle{empty}
    \clearpage
    \addtocounter{page}{-1}
}

% urlのフォントを直す
\renewcommand\UrlFont{\rmfamily}
